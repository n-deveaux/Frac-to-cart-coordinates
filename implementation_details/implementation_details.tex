% !TeX spellcheck = en_US
\documentclass[12pt,a4paper,oneside]{article}

\usepackage{setspace}
\onehalfspacing

\usepackage[utf8]{inputenc}
\usepackage[english]{babel}
\usepackage{graphicx}
\usepackage{lmodern}
\usepackage[left=2.0cm,right=2.0cm,top=2.2cm,bottom=2.2cm]{geometry}
\usepackage[T1]{fontenc}
\usepackage[cyr]{aeguill}

\usepackage{epsfig}

\usepackage{amsmath,amsthm}
\usepackage{amsfonts,amssymb}
\usepackage{url}

\usepackage[rightcaption]{sidecap}
\usepackage{caption}

\usepackage{array}

\usepackage[articletitle = true, etalmode=truncate, maxauthors=10]{achemso}
\usepackage[colorlinks=true,linkcolor=black,urlcolor=black,breaklinks=true,citecolor=blue]{hyperref} 

%math gras en itallique
\usepackage{bm}

\begin{document}
	 \begin{center}
		\LARGE{\textbf{Frac-to-cart-coordinates implementation aspects}}\\
		\vspace{1cm}
	\end{center}

\section*{Coordinate transformation in crystallography}
In crystallography, atomic positions within a crystal's unit cell can be described using:
\begin{itemize}
	\item \textbf{Fractional coordinates} $(x/a, y/b, z/c)$, which refer to the natural axes \(a, b, c\), scaled by their respective unit cell lengths.
	
	\item \textbf{Orthogonal coordinates} $(X, Y, Z)$, which use a right-angled Cartesian system with distances measured in Ångstroms.
	
\end{itemize}
For triclinic unit cells, the relationship between these coordinate systems involves a transformation matrix with non-trivial elements.

\section*{Transformation matrix}
This implementation follows the methodology described on Jon Cooper's website\cite{Cooper-site}. It relies on the fundamental principles of spherical trigonometry, allowing the cosine rule for reciprocal angles:
\begin{equation}
	\cos(\alpha^*) = \frac{\cos(\beta)\cos(\gamma) - \cos(\alpha)}{\sin(\beta)\sin(\gamma)}.
\end{equation}
Using this, the transformation matrix to convert Cartesian orthogonal coordinates $(X, Y, Z)$ to fractional ones $(x, y, z)$ aligning $a$ with the $X$-axis, is:	
\begin{equation}
	\begin{pmatrix}
		X \\
		Y \\
		Z
	\end{pmatrix}
	=
	\begin{pmatrix}
		a & b\cos(\gamma) & c\cos(\beta) \\
		0 & b\sin(\gamma) & -c\sin(\beta)\cos(\alpha^*) \\
		0 & 0 & c\sin(\beta)\sin(\alpha^*)
	\end{pmatrix}
	\begin{pmatrix}
		x \\
		y \\
		z
	\end{pmatrix}
\end{equation}
where $a, b, c$ are the unit cell lengths, $\alpha, \beta, \gamma$ are the angles of the unit cell (in radians), and $\alpha^*$ is the reciprocal angle, calculated using the cosine rule above.

	\bibliography{implementation_details-refs}
\end{document}